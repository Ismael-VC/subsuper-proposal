\documentclass[10pt,english]{article}
\usepackage[T1]{fontenc}
\usepackage[utf8]{inputenc}
\usepackage{geometry}
\geometry{verbose,lmargin=1in,rmargin=1in}
\usepackage{babel}
\usepackage{mathrsfs}
\usepackage{amsbsy}
\usepackage{amssymb}
\usepackage[unicode=true]
 {hyperref}

\makeatletter

\usepackage{upgreek}
\usepackage{cite}

\makeatother

\begin{document}

\title{Proposal to Encode Subscripts/Superscripts\\
for Mathematical Programming}


\author{Prof. Steven G. Johnson, MIT Department of Mathematics }

\maketitle
\begin{center}
.... add other endorsers here --- we don't want to turn this into
an Internet petition, but endorsements from well-known developers,
organizations, open-source projects, and companies, could be helpful
\par\end{center}


\section{Summary}

Many recent programming languages (including Python 3~\cite{Python},
Go~\cite{Go}, Julia~\cite{Julia}, Fortress~\cite{Fortress},
Java~\cite{Java}, C\#~\cite{Csharp}, and Swift\cite{Swift}) allow
a broad range of Unicode characters to be employed by the programmer
for identifiers (names of functions and variables), and indeed such
behavior is recommended by the \emph{Unicode Standard Annex~\#31}~\cite{UAX31}.
For programmers working in mathematical fields, including statistics,
science, and engineering, where a wide array of mathematical notations
and symbols are commonplace, this often enables computer programs
to be written in a notation that closely mimics their standard mathematical
descriptions. For example, in a mathematical context one might easily
have a quantity with a name like $\hat{x}$ or $\alpha_{x}$, and
the same names can now be used in computer programs for this quantity
($\mathrm{\hat{{x}}}$ = U+0078 U+0302, $\mathrm{{\upalpha_{x}}}$
= U+03B1 U+2093) rather than ASCII approximations such as \texttt{xhat}
and \texttt{alpha\_x} that are used in older programming languages.
Furthermore, text-editing software is making it increasingly easy
for programmers to type such symbols by ``tab completion''
(e.g. typing ``$\mathrm{{\upalpha_{x}}}$''
as \texttt{\textbackslash{}alpha}\texttt{\emph{{[}tab{]}}}\texttt{\textbackslash{}\_x}\texttt{\emph{{[}tab{]}}})
or related techniques in several popular editors~\cite{Atom,Emacs,Sublime,Vim}
and interactive programming environments~\cite{Julia,IPython}. This
is an important and positive trend. To quote a 2001 draft of \emph{Unicode
Technical Report~\#25}~\cite[section 5.3]{UTR25}, \emph{Using real
mathematical expressions in computer programs would be far superior
in terms of readability, reduced coding times, program maintenance,
and streamlined documentation.}

One notation that is extremely common throughout all of mathematics
is the use of subscripts and superscripts. Unfortunately, Unicode
9.0 provides only a very limited selection of subscript and superscript
characters (digits 0--9, parentheses, $+$/$-$/$=$, a subset of
the Latin and Greek alphabets, and a few phonetic symbols). Therefore,\textbf{
we propose that Unicode be extended to encode all the subscript/superscript
characters most commonly used in mathematical notation}: all lower-
and upper-case Latin and Greek letters (and variants thereof) and
a selection of other mathematical symbols described below. Because
these are all derived from existing Unicode codepoints, the assignment
of their properties and the development of their fonts will be straightforward.
\textbf{Alternatively}, we urge the Consortium to consider an approach
that is more elegant and general, but whose implementation is slightly
more ambitious: \textbf{add new ``mathematical subscript'' and ``mathematical
superscript'' combining characters} that convert the preceding glyph
(character + combining characters) into a sub/superscript. (This is
closely related to the superscript and ``scientific
inferiors'' features already present in OpenType~\cite{OpenType}.)
Note that the two approaches are not incompatible---a more limited
set of sub/superscripts implemented now could co-exist with combining
marks implemented later, via canonical decomposition/composition to
convert between them.

We should mention that there were past proposals to extend the range
of subscript and superscript characters in Unicode~\cite{L2-10-230,L2-11-208}
(neither of which mentioned mathematical or programming applications)
that were rejected. We have been unable to find an official record
of the grounds for rejection of these proposals, but there seem to
be two common objections in discussions of this topic~\cite{Miller10,UCDF}.
First, that this is mere ``formatting''
devoid of semantic content, and/or is better accomplished via markup
languages like MathML or other typesetting technologies. Second, that
the set of characters that deserve sub/superscript variants is unclear
and might be subjected to never-ending extension. We believe that
both of those objections must be re-evaluated in light of mathematical
programming. First, subscripts and superscripts of any kind in mathematics
always have semantic content ($\mathrm{{\upalpha_{x}}}$ is very different
in meaning from $\mathrm{\upalpha\mathrm{{x}}}$). Second, \emph{programming
languages are almost invariably restricted to ``plain text''}---in
ordinary usage, computer code (and especially identifiers) cannot
be intermingled with MathML markup or other formatting codes---with
a few exceptions such as Mathematica~\cite{Mathematica} that are
edited and/or rendered mainly within specialized software). Third,
new subscript/superscript characters are explicitly enumerated (as
opposed to defining new combining characters), the restriction to
\emph{common mathematical notations} immediately limits the scope
of the additions, excluding the vast majority of Unicode characters
from consideration.

While it is probably not practical to encode the full range of mathematical
notations in Unicode---MathML and similar will always be useful---subscripts
and superscripts are low-hanging fruit that would be tremendously
beneficial to computer programming in technical fields and are already
partially present in Unicode. We respectfully urge you to expand the
set of mathematical subscripts and superscripts, and especially to
eliminate frustrating omissions from the Latin and Greek alphabets.


\section{Proposed Additions to Subscript/Superscript Characters}

All lowercase and uppercase Latin and Greek letters.

Should we also ask for the italic ($\mathit{{X}}$), bold ($\mathbf{{X}}$),
and bold-italic ($\mathbf{\mathit{\boldsymbol{{X}}}}$) variants?
Script variants like $\mathscr{X}$, and blackboard-bold variants
like $\mathbb{{X}}$? Or is this asking for too much?

What other math symbols are commonly used as super/subscripts? $\Vert,\perp,*,\dagger,\ldots$
?

We also could use punctuation: commas and semicolons, for things like
$A_{i,j}$. This is a bit tricky because most programming languages
do not allow punctuation in identifiers (although Julia allows super/subscript
parentheses). 


\subsection{Character Properties}

(Case, general category, collation order, names, bidirectional class
and mirrored properties, simple uppercase/lowercase mapping, linebreaking,
use in identifiers. In general, these will be identical to those of
the existing Unicode character that we are modifying into super/subscript
form.)


\section{Generalized Alternative: New Combining Characters}


\section{Lack of Possible Equivalents}

``Ensure that documentation supporting the proposal states whether
any Unicode characters were examined as possible equivalents for the
proposed character and, if so, why each was rejected.'' (\url{http://unicode.org/pending/proposals.html})


\section{Provision of Fonts}

Proposals should include ``the name and contact information
for a company or individual who would agree to provide a computerized
font (True Type or PostScript) for publication of the standard''
(\url{http://unicode.org/pending/proposals.html})
\dots{} how can we do this?


\section{Contact Information}

\emph{``names and addresses of appropriate contacts
within national body or user organizations''}
\begin{itemize}
\item Steven G. Johnson, MIT Room 2-345, 77 Massachusetts Ave. Cambridge,
MA 02139\end{itemize}
\begin{thebibliography}{10}
\bibitem{Python}Martin von L{\"o}wis, ``Python PEP 3131:
Supporting Non-ASCII Identifiers,'' \url{https://www.python.org/dev/peps/pep-3131/}
(2007-05-01).

\bibitem{Go}The Go Authors, \emph{The Go Programming Language Specification},
\url{https://golang.org/ref/spec} (2016-05-31).

\bibitem{Julia}The Julia Authors, \emph{The Julia Programming Language
Manual}, \url{http://docs.julialang.org/en/latest/manual/variables/}
(retrieved 2016-08-25).

\bibitem{Fortress}Sun Microsystems, Inc., \emph{The Fortress Language
Specification}, version 1.0, \url{http://www.ccs.neu.edu/home/samth/fortress-spec.pdf}
(2008-03-31).

\bibitem{Java}James Gosling, Bill Joy, Guy Steele, Gilad Bracha,
and Alex Buckley, \emph{The Java Language Specification: Java SE 8
Edition}, section 3.8, \url{http://docs.oracle.com/javase/specs/jls/se8/html/jls-3.html\#jls-3.8}
(2015-02-13).

\bibitem{Csharp}Microsoft Corporation, \emph{C\# Language Specification},
version 5.0, section 2.4.2 (2012).

\bibitem{Swift}Apple Inc, \emph{The Swift Programming Language},
version 2.2, \url{https://developer.apple.com/library/ios/documentation/Swift/Conceptual/Swift_Programming_Language/LexicalStructure.html}
(2016-03-21).

\bibitem{UAX31}Mark Davis, \emph{Unicode Standard Annex \#31: Unicode
Identifier and Pattern Syntax}, version 9.0.0, \url{http://www.unicode.org/reports/tr31/tr31-25.html}
(2016-05-31).

\bibitem{Atom}atom-latex-completions plugin for the Atom editor,
\url{https://github.com/JunoLab/atom-latex-completions}
(retrieved 2016-08-25).

\bibitem{Emacs}\TeX{} Input Method for the Emacs editor, \url{https://www.emacswiki.org/emacs/TeXInputMethod}
(retrieved 2016-08-25).

\bibitem{Sublime}UnicodeMath plugin for the Sublime editor, \url{https://github.com/mvoidex/UnicodeMath}
(retrieved 2016-08-25).

\bibitem{Vim}julia-vim plugin for the vim editor, \url{https://github.com/JuliaLang/julia-vim}
(retrieved 2016-08-25).

\bibitem{IPython}Brian E. Granger, ``Adds Julia-style
latex->unicode tab completion,'' \emph{IPython pull
request \#6380}, \url{https://github.com/ipython/ipython/pull/6380}
(2014-08-28).

\bibitem{UTR25}Barbara Beeton, Asmus Freytag, and Murray Sargent
III, ``Proposed Draft Unicode Technical Report \#25:
Unicode Support for Mathematics'' \url{http://www.unicode.org/unicode/reports/tr25/tr25-1.html} (2001-10-10).

\bibitem{L2-10-230}Karl Pentzlin, ``Proposal to encode
a modifier letter used in French abbreviations in the UCS,''
\url{http://www.unicode.org/L2/L2010/10230-modifier-q.pdf}
(2010-07-13). 

\bibitem{L2-11-208}ISO/IEC JTC1/SC35/WG1, ``Proposal
to encode missing Latin small capital and modifier letters in the
UCS,'' \url{http://www.unicode.org/L2/L2011/11208-n4068.pdf}
(2011-03-12).

\bibitem{Miller10}Eric Miller, ``Comment on L2-10-230,
Proposal to encode a modifier letter used in French abbreviations
in the UCS,'' \url{http://www.unicode.org/L2/L2010/10315-comment.pdf}
(2010-08-09).

\bibitem{UCDF}``Superscript latin small letter q,''
discussion on \emph{The Unicode Consortium Discussion Forum}, \url{http://www.unicode.org/forum/viewtopic.php?f=24&t=142}
(2011).

\bibitem{Mathematica}Stephen Wolfram, ``Mathematical
Notation: Past and Future,'' \url{http://www.stephenwolfram.com/publications/mathematical-notation-past-future/}
(2000).

\bibitem{OpenType}Microsoft Corporation, ``Registered
Features,'' \emph{OpenType Specification}, version
1.7 \url{https://www.microsoft.com/typography/otspec/features_pt.htm\#sinf}
(2008-11-19). \end{thebibliography}

\end{document}
